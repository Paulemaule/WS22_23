
% Default to the notebook output style

    


% Inherit from the specified cell style.




    
\documentclass[11pt]{article}

    
    
    \usepackage[T1]{fontenc}
    % Nicer default font (+ math font) than Computer Modern for most use cases
    \usepackage{mathpazo}

    % Basic figure setup, for now with no caption control since it's done
    % automatically by Pandoc (which extracts ![](path) syntax from Markdown).
    \usepackage{graphicx}
    % We will generate all images so they have a width \maxwidth. This means
    % that they will get their normal width if they fit onto the page, but
    % are scaled down if they would overflow the margins.
    \makeatletter
    \def\maxwidth{\ifdim\Gin@nat@width>\linewidth\linewidth
    \else\Gin@nat@width\fi}
    \makeatother
    \let\Oldincludegraphics\includegraphics
    % Set max figure width to be 80% of text width, for now hardcoded.
    \renewcommand{\includegraphics}[1]{\Oldincludegraphics[width=.8\maxwidth]{#1}}
    % Ensure that by default, figures have no caption (until we provide a
    % proper Figure object with a Caption API and a way to capture that
    % in the conversion process - todo).
    \usepackage{caption}
    \DeclareCaptionLabelFormat{nolabel}{}
    \captionsetup{labelformat=nolabel}

    \usepackage{adjustbox} % Used to constrain images to a maximum size 
    \usepackage{xcolor} % Allow colors to be defined
    \usepackage{enumerate} % Needed for markdown enumerations to work
    \usepackage{geometry} % Used to adjust the document margins
    \usepackage{amsmath} % Equations
    \usepackage{amssymb} % Equations
    \usepackage{textcomp} % defines textquotesingle
    % Hack from http://tex.stackexchange.com/a/47451/13684:
    \AtBeginDocument{%
        \def\PYZsq{\textquotesingle}% Upright quotes in Pygmentized code
    }
    \usepackage{upquote} % Upright quotes for verbatim code
    \usepackage{eurosym} % defines \euro
    \usepackage[mathletters]{ucs} % Extended unicode (utf-8) support
    \usepackage[utf8x]{inputenc} % Allow utf-8 characters in the tex document
    \usepackage{fancyvrb} % verbatim replacement that allows latex
    \usepackage{grffile} % extends the file name processing of package graphics 
                         % to support a larger range 
    % The hyperref package gives us a pdf with properly built
    % internal navigation ('pdf bookmarks' for the table of contents,
    % internal cross-reference links, web links for URLs, etc.)
    \usepackage{hyperref}
    \usepackage{longtable} % longtable support required by pandoc >1.10
    \usepackage{booktabs}  % table support for pandoc > 1.12.2
    \usepackage[inline]{enumitem} % IRkernel/repr support (it uses the enumerate* environment)
    \usepackage[normalem]{ulem} % ulem is needed to support strikethroughs (\sout)
                                % normalem makes italics be italics, not underlines
    

    
    
    % Colors for the hyperref package
    \definecolor{urlcolor}{rgb}{0,.145,.698}
    \definecolor{linkcolor}{rgb}{.71,0.21,0.01}
    \definecolor{citecolor}{rgb}{.12,.54,.11}

    % ANSI colors
    \definecolor{ansi-black}{HTML}{3E424D}
    \definecolor{ansi-black-intense}{HTML}{282C36}
    \definecolor{ansi-red}{HTML}{E75C58}
    \definecolor{ansi-red-intense}{HTML}{B22B31}
    \definecolor{ansi-green}{HTML}{00A250}
    \definecolor{ansi-green-intense}{HTML}{007427}
    \definecolor{ansi-yellow}{HTML}{DDB62B}
    \definecolor{ansi-yellow-intense}{HTML}{B27D12}
    \definecolor{ansi-blue}{HTML}{208FFB}
    \definecolor{ansi-blue-intense}{HTML}{0065CA}
    \definecolor{ansi-magenta}{HTML}{D160C4}
    \definecolor{ansi-magenta-intense}{HTML}{A03196}
    \definecolor{ansi-cyan}{HTML}{60C6C8}
    \definecolor{ansi-cyan-intense}{HTML}{258F8F}
    \definecolor{ansi-white}{HTML}{C5C1B4}
    \definecolor{ansi-white-intense}{HTML}{A1A6B2}

    % commands and environments needed by pandoc snippets
    % extracted from the output of `pandoc -s`
    \providecommand{\tightlist}{%
      \setlength{\itemsep}{0pt}\setlength{\parskip}{0pt}}
    \DefineVerbatimEnvironment{Highlighting}{Verbatim}{commandchars=\\\{\}}
    % Add ',fontsize=\small' for more characters per line
    \newenvironment{Shaded}{}{}
    \newcommand{\KeywordTok}[1]{\textcolor[rgb]{0.00,0.44,0.13}{\textbf{{#1}}}}
    \newcommand{\DataTypeTok}[1]{\textcolor[rgb]{0.56,0.13,0.00}{{#1}}}
    \newcommand{\DecValTok}[1]{\textcolor[rgb]{0.25,0.63,0.44}{{#1}}}
    \newcommand{\BaseNTok}[1]{\textcolor[rgb]{0.25,0.63,0.44}{{#1}}}
    \newcommand{\FloatTok}[1]{\textcolor[rgb]{0.25,0.63,0.44}{{#1}}}
    \newcommand{\CharTok}[1]{\textcolor[rgb]{0.25,0.44,0.63}{{#1}}}
    \newcommand{\StringTok}[1]{\textcolor[rgb]{0.25,0.44,0.63}{{#1}}}
    \newcommand{\CommentTok}[1]{\textcolor[rgb]{0.38,0.63,0.69}{\textit{{#1}}}}
    \newcommand{\OtherTok}[1]{\textcolor[rgb]{0.00,0.44,0.13}{{#1}}}
    \newcommand{\AlertTok}[1]{\textcolor[rgb]{1.00,0.00,0.00}{\textbf{{#1}}}}
    \newcommand{\FunctionTok}[1]{\textcolor[rgb]{0.02,0.16,0.49}{{#1}}}
    \newcommand{\RegionMarkerTok}[1]{{#1}}
    \newcommand{\ErrorTok}[1]{\textcolor[rgb]{1.00,0.00,0.00}{\textbf{{#1}}}}
    \newcommand{\NormalTok}[1]{{#1}}
    
    % Additional commands for more recent versions of Pandoc
    \newcommand{\ConstantTok}[1]{\textcolor[rgb]{0.53,0.00,0.00}{{#1}}}
    \newcommand{\SpecialCharTok}[1]{\textcolor[rgb]{0.25,0.44,0.63}{{#1}}}
    \newcommand{\VerbatimStringTok}[1]{\textcolor[rgb]{0.25,0.44,0.63}{{#1}}}
    \newcommand{\SpecialStringTok}[1]{\textcolor[rgb]{0.73,0.40,0.53}{{#1}}}
    \newcommand{\ImportTok}[1]{{#1}}
    \newcommand{\DocumentationTok}[1]{\textcolor[rgb]{0.73,0.13,0.13}{\textit{{#1}}}}
    \newcommand{\AnnotationTok}[1]{\textcolor[rgb]{0.38,0.63,0.69}{\textbf{\textit{{#1}}}}}
    \newcommand{\CommentVarTok}[1]{\textcolor[rgb]{0.38,0.63,0.69}{\textbf{\textit{{#1}}}}}
    \newcommand{\VariableTok}[1]{\textcolor[rgb]{0.10,0.09,0.49}{{#1}}}
    \newcommand{\ControlFlowTok}[1]{\textcolor[rgb]{0.00,0.44,0.13}{\textbf{{#1}}}}
    \newcommand{\OperatorTok}[1]{\textcolor[rgb]{0.40,0.40,0.40}{{#1}}}
    \newcommand{\BuiltInTok}[1]{{#1}}
    \newcommand{\ExtensionTok}[1]{{#1}}
    \newcommand{\PreprocessorTok}[1]{\textcolor[rgb]{0.74,0.48,0.00}{{#1}}}
    \newcommand{\AttributeTok}[1]{\textcolor[rgb]{0.49,0.56,0.16}{{#1}}}
    \newcommand{\InformationTok}[1]{\textcolor[rgb]{0.38,0.63,0.69}{\textbf{\textit{{#1}}}}}
    \newcommand{\WarningTok}[1]{\textcolor[rgb]{0.38,0.63,0.69}{\textbf{\textit{{#1}}}}}
    
    
    % Define a nice break command that doesn't care if a line doesn't already
    % exist.
    \def\br{\hspace*{\fill} \\* }
    % Math Jax compatability definitions
    \def\gt{>}
    \def\lt{<}
    % Document parameters
    \title{sheet01}
    
    
    

    % Pygments definitions
    
\makeatletter
\def\PY@reset{\let\PY@it=\relax \let\PY@bf=\relax%
    \let\PY@ul=\relax \let\PY@tc=\relax%
    \let\PY@bc=\relax \let\PY@ff=\relax}
\def\PY@tok#1{\csname PY@tok@#1\endcsname}
\def\PY@toks#1+{\ifx\relax#1\empty\else%
    \PY@tok{#1}\expandafter\PY@toks\fi}
\def\PY@do#1{\PY@bc{\PY@tc{\PY@ul{%
    \PY@it{\PY@bf{\PY@ff{#1}}}}}}}
\def\PY#1#2{\PY@reset\PY@toks#1+\relax+\PY@do{#2}}

\expandafter\def\csname PY@tok@w\endcsname{\def\PY@tc##1{\textcolor[rgb]{0.73,0.73,0.73}{##1}}}
\expandafter\def\csname PY@tok@c\endcsname{\let\PY@it=\textit\def\PY@tc##1{\textcolor[rgb]{0.25,0.50,0.50}{##1}}}
\expandafter\def\csname PY@tok@cp\endcsname{\def\PY@tc##1{\textcolor[rgb]{0.74,0.48,0.00}{##1}}}
\expandafter\def\csname PY@tok@k\endcsname{\let\PY@bf=\textbf\def\PY@tc##1{\textcolor[rgb]{0.00,0.50,0.00}{##1}}}
\expandafter\def\csname PY@tok@kp\endcsname{\def\PY@tc##1{\textcolor[rgb]{0.00,0.50,0.00}{##1}}}
\expandafter\def\csname PY@tok@kt\endcsname{\def\PY@tc##1{\textcolor[rgb]{0.69,0.00,0.25}{##1}}}
\expandafter\def\csname PY@tok@o\endcsname{\def\PY@tc##1{\textcolor[rgb]{0.40,0.40,0.40}{##1}}}
\expandafter\def\csname PY@tok@ow\endcsname{\let\PY@bf=\textbf\def\PY@tc##1{\textcolor[rgb]{0.67,0.13,1.00}{##1}}}
\expandafter\def\csname PY@tok@nb\endcsname{\def\PY@tc##1{\textcolor[rgb]{0.00,0.50,0.00}{##1}}}
\expandafter\def\csname PY@tok@nf\endcsname{\def\PY@tc##1{\textcolor[rgb]{0.00,0.00,1.00}{##1}}}
\expandafter\def\csname PY@tok@nc\endcsname{\let\PY@bf=\textbf\def\PY@tc##1{\textcolor[rgb]{0.00,0.00,1.00}{##1}}}
\expandafter\def\csname PY@tok@nn\endcsname{\let\PY@bf=\textbf\def\PY@tc##1{\textcolor[rgb]{0.00,0.00,1.00}{##1}}}
\expandafter\def\csname PY@tok@ne\endcsname{\let\PY@bf=\textbf\def\PY@tc##1{\textcolor[rgb]{0.82,0.25,0.23}{##1}}}
\expandafter\def\csname PY@tok@nv\endcsname{\def\PY@tc##1{\textcolor[rgb]{0.10,0.09,0.49}{##1}}}
\expandafter\def\csname PY@tok@no\endcsname{\def\PY@tc##1{\textcolor[rgb]{0.53,0.00,0.00}{##1}}}
\expandafter\def\csname PY@tok@nl\endcsname{\def\PY@tc##1{\textcolor[rgb]{0.63,0.63,0.00}{##1}}}
\expandafter\def\csname PY@tok@ni\endcsname{\let\PY@bf=\textbf\def\PY@tc##1{\textcolor[rgb]{0.60,0.60,0.60}{##1}}}
\expandafter\def\csname PY@tok@na\endcsname{\def\PY@tc##1{\textcolor[rgb]{0.49,0.56,0.16}{##1}}}
\expandafter\def\csname PY@tok@nt\endcsname{\let\PY@bf=\textbf\def\PY@tc##1{\textcolor[rgb]{0.00,0.50,0.00}{##1}}}
\expandafter\def\csname PY@tok@nd\endcsname{\def\PY@tc##1{\textcolor[rgb]{0.67,0.13,1.00}{##1}}}
\expandafter\def\csname PY@tok@s\endcsname{\def\PY@tc##1{\textcolor[rgb]{0.73,0.13,0.13}{##1}}}
\expandafter\def\csname PY@tok@sd\endcsname{\let\PY@it=\textit\def\PY@tc##1{\textcolor[rgb]{0.73,0.13,0.13}{##1}}}
\expandafter\def\csname PY@tok@si\endcsname{\let\PY@bf=\textbf\def\PY@tc##1{\textcolor[rgb]{0.73,0.40,0.53}{##1}}}
\expandafter\def\csname PY@tok@se\endcsname{\let\PY@bf=\textbf\def\PY@tc##1{\textcolor[rgb]{0.73,0.40,0.13}{##1}}}
\expandafter\def\csname PY@tok@sr\endcsname{\def\PY@tc##1{\textcolor[rgb]{0.73,0.40,0.53}{##1}}}
\expandafter\def\csname PY@tok@ss\endcsname{\def\PY@tc##1{\textcolor[rgb]{0.10,0.09,0.49}{##1}}}
\expandafter\def\csname PY@tok@sx\endcsname{\def\PY@tc##1{\textcolor[rgb]{0.00,0.50,0.00}{##1}}}
\expandafter\def\csname PY@tok@m\endcsname{\def\PY@tc##1{\textcolor[rgb]{0.40,0.40,0.40}{##1}}}
\expandafter\def\csname PY@tok@gh\endcsname{\let\PY@bf=\textbf\def\PY@tc##1{\textcolor[rgb]{0.00,0.00,0.50}{##1}}}
\expandafter\def\csname PY@tok@gu\endcsname{\let\PY@bf=\textbf\def\PY@tc##1{\textcolor[rgb]{0.50,0.00,0.50}{##1}}}
\expandafter\def\csname PY@tok@gd\endcsname{\def\PY@tc##1{\textcolor[rgb]{0.63,0.00,0.00}{##1}}}
\expandafter\def\csname PY@tok@gi\endcsname{\def\PY@tc##1{\textcolor[rgb]{0.00,0.63,0.00}{##1}}}
\expandafter\def\csname PY@tok@gr\endcsname{\def\PY@tc##1{\textcolor[rgb]{1.00,0.00,0.00}{##1}}}
\expandafter\def\csname PY@tok@ge\endcsname{\let\PY@it=\textit}
\expandafter\def\csname PY@tok@gs\endcsname{\let\PY@bf=\textbf}
\expandafter\def\csname PY@tok@gp\endcsname{\let\PY@bf=\textbf\def\PY@tc##1{\textcolor[rgb]{0.00,0.00,0.50}{##1}}}
\expandafter\def\csname PY@tok@go\endcsname{\def\PY@tc##1{\textcolor[rgb]{0.53,0.53,0.53}{##1}}}
\expandafter\def\csname PY@tok@gt\endcsname{\def\PY@tc##1{\textcolor[rgb]{0.00,0.27,0.87}{##1}}}
\expandafter\def\csname PY@tok@err\endcsname{\def\PY@bc##1{\setlength{\fboxsep}{0pt}\fcolorbox[rgb]{1.00,0.00,0.00}{1,1,1}{\strut ##1}}}
\expandafter\def\csname PY@tok@kc\endcsname{\let\PY@bf=\textbf\def\PY@tc##1{\textcolor[rgb]{0.00,0.50,0.00}{##1}}}
\expandafter\def\csname PY@tok@kd\endcsname{\let\PY@bf=\textbf\def\PY@tc##1{\textcolor[rgb]{0.00,0.50,0.00}{##1}}}
\expandafter\def\csname PY@tok@kn\endcsname{\let\PY@bf=\textbf\def\PY@tc##1{\textcolor[rgb]{0.00,0.50,0.00}{##1}}}
\expandafter\def\csname PY@tok@kr\endcsname{\let\PY@bf=\textbf\def\PY@tc##1{\textcolor[rgb]{0.00,0.50,0.00}{##1}}}
\expandafter\def\csname PY@tok@bp\endcsname{\def\PY@tc##1{\textcolor[rgb]{0.00,0.50,0.00}{##1}}}
\expandafter\def\csname PY@tok@fm\endcsname{\def\PY@tc##1{\textcolor[rgb]{0.00,0.00,1.00}{##1}}}
\expandafter\def\csname PY@tok@vc\endcsname{\def\PY@tc##1{\textcolor[rgb]{0.10,0.09,0.49}{##1}}}
\expandafter\def\csname PY@tok@vg\endcsname{\def\PY@tc##1{\textcolor[rgb]{0.10,0.09,0.49}{##1}}}
\expandafter\def\csname PY@tok@vi\endcsname{\def\PY@tc##1{\textcolor[rgb]{0.10,0.09,0.49}{##1}}}
\expandafter\def\csname PY@tok@vm\endcsname{\def\PY@tc##1{\textcolor[rgb]{0.10,0.09,0.49}{##1}}}
\expandafter\def\csname PY@tok@sa\endcsname{\def\PY@tc##1{\textcolor[rgb]{0.73,0.13,0.13}{##1}}}
\expandafter\def\csname PY@tok@sb\endcsname{\def\PY@tc##1{\textcolor[rgb]{0.73,0.13,0.13}{##1}}}
\expandafter\def\csname PY@tok@sc\endcsname{\def\PY@tc##1{\textcolor[rgb]{0.73,0.13,0.13}{##1}}}
\expandafter\def\csname PY@tok@dl\endcsname{\def\PY@tc##1{\textcolor[rgb]{0.73,0.13,0.13}{##1}}}
\expandafter\def\csname PY@tok@s2\endcsname{\def\PY@tc##1{\textcolor[rgb]{0.73,0.13,0.13}{##1}}}
\expandafter\def\csname PY@tok@sh\endcsname{\def\PY@tc##1{\textcolor[rgb]{0.73,0.13,0.13}{##1}}}
\expandafter\def\csname PY@tok@s1\endcsname{\def\PY@tc##1{\textcolor[rgb]{0.73,0.13,0.13}{##1}}}
\expandafter\def\csname PY@tok@mb\endcsname{\def\PY@tc##1{\textcolor[rgb]{0.40,0.40,0.40}{##1}}}
\expandafter\def\csname PY@tok@mf\endcsname{\def\PY@tc##1{\textcolor[rgb]{0.40,0.40,0.40}{##1}}}
\expandafter\def\csname PY@tok@mh\endcsname{\def\PY@tc##1{\textcolor[rgb]{0.40,0.40,0.40}{##1}}}
\expandafter\def\csname PY@tok@mi\endcsname{\def\PY@tc##1{\textcolor[rgb]{0.40,0.40,0.40}{##1}}}
\expandafter\def\csname PY@tok@il\endcsname{\def\PY@tc##1{\textcolor[rgb]{0.40,0.40,0.40}{##1}}}
\expandafter\def\csname PY@tok@mo\endcsname{\def\PY@tc##1{\textcolor[rgb]{0.40,0.40,0.40}{##1}}}
\expandafter\def\csname PY@tok@ch\endcsname{\let\PY@it=\textit\def\PY@tc##1{\textcolor[rgb]{0.25,0.50,0.50}{##1}}}
\expandafter\def\csname PY@tok@cm\endcsname{\let\PY@it=\textit\def\PY@tc##1{\textcolor[rgb]{0.25,0.50,0.50}{##1}}}
\expandafter\def\csname PY@tok@cpf\endcsname{\let\PY@it=\textit\def\PY@tc##1{\textcolor[rgb]{0.25,0.50,0.50}{##1}}}
\expandafter\def\csname PY@tok@c1\endcsname{\let\PY@it=\textit\def\PY@tc##1{\textcolor[rgb]{0.25,0.50,0.50}{##1}}}
\expandafter\def\csname PY@tok@cs\endcsname{\let\PY@it=\textit\def\PY@tc##1{\textcolor[rgb]{0.25,0.50,0.50}{##1}}}

\def\PYZbs{\char`\\}
\def\PYZus{\char`\_}
\def\PYZob{\char`\{}
\def\PYZcb{\char`\}}
\def\PYZca{\char`\^}
\def\PYZam{\char`\&}
\def\PYZlt{\char`\<}
\def\PYZgt{\char`\>}
\def\PYZsh{\char`\#}
\def\PYZpc{\char`\%}
\def\PYZdl{\char`\$}
\def\PYZhy{\char`\-}
\def\PYZsq{\char`\'}
\def\PYZdq{\char`\"}
\def\PYZti{\char`\~}
% for compatibility with earlier versions
\def\PYZat{@}
\def\PYZlb{[}
\def\PYZrb{]}
\makeatother


    % Exact colors from NB
    \definecolor{incolor}{rgb}{0.0, 0.0, 0.5}
    \definecolor{outcolor}{rgb}{0.545, 0.0, 0.0}



    
    % Prevent overflowing lines due to hard-to-break entities
    \sloppy 
    % Setup hyperref package
    \hypersetup{
      breaklinks=true,  % so long urls are correctly broken across lines
      colorlinks=true,
      urlcolor=urlcolor,
      linkcolor=linkcolor,
      citecolor=citecolor,
      }
    % Slightly bigger margins than the latex defaults
    
    \geometry{verbose,tmargin=1in,bmargin=1in,lmargin=1in,rmargin=1in}
    
    

    \begin{document}
    
    
    \maketitle
    
    

    
    \section{Sheet 1}\label{sheet-1}

    \begin{Verbatim}[commandchars=\\\{\}]
{\color{incolor}In [{\color{incolor}2}]:} \PY{k+kn}{import} \PY{n+nn}{numpy} \PY{k}{as} \PY{n+nn}{np}
        \PY{k+kn}{import} \PY{n+nn}{matplotlib}\PY{n+nn}{.}\PY{n+nn}{pyplot} \PY{k}{as} \PY{n+nn}{plt} 
        \PY{k+kn}{from} \PY{n+nn}{matplotlib} \PY{k}{import} \PY{n}{pyplot} \PY{k}{as} \PY{n}{plt}
        \PY{o}{\PYZpc{}}\PY{k}{matplotlib} inline
\end{Verbatim}


    \subsection{1 Principal Component
Analysis}\label{principal-component-analysis}

\subsubsection{(a)}\label{a}

    \begin{Verbatim}[commandchars=\\\{\}]
{\color{incolor}In [{\color{incolor}3}]:} \PY{c+c1}{\PYZsh{} TODO: implement PCA (fill in the blanks in the function below)}
        
        \PY{k}{def} \PY{n+nf}{pca}\PY{p}{(}\PY{n}{data}\PY{p}{,} \PY{n}{n\PYZus{}components}\PY{o}{=}\PY{k+kc}{None}\PY{p}{)}\PY{p}{:}
            \PY{l+s+sd}{\PYZdq{}\PYZdq{}\PYZdq{}}
        \PY{l+s+sd}{    Principal Component Analysis on a p x N data matrix.}
        \PY{l+s+sd}{    }
        \PY{l+s+sd}{    Parameters}
        \PY{l+s+sd}{    \PYZhy{}\PYZhy{}\PYZhy{}\PYZhy{}\PYZhy{}\PYZhy{}\PYZhy{}\PYZhy{}\PYZhy{}\PYZhy{}}
        \PY{l+s+sd}{    data : np.ndarray}
        \PY{l+s+sd}{        Data matrix of shape (p, N).}
        \PY{l+s+sd}{    n\PYZus{}components : int, optional}
        \PY{l+s+sd}{        Number of requested components. By default returns all components.}
        \PY{l+s+sd}{        }
        \PY{l+s+sd}{    Returns}
        \PY{l+s+sd}{    \PYZhy{}\PYZhy{}\PYZhy{}\PYZhy{}\PYZhy{}\PYZhy{}\PYZhy{}}
        \PY{l+s+sd}{    np.ndarray, np.ndarray}
        \PY{l+s+sd}{        the pca components (shape (n\PYZus{}components, p)) and the projection (shape (n\PYZus{}components, N))}
        
        \PY{l+s+sd}{    \PYZdq{}\PYZdq{}\PYZdq{}}
            \PY{c+c1}{\PYZsh{} set n\PYZus{}components to p by default}
            \PY{n}{n\PYZus{}components} \PY{o}{=} \PY{n}{data}\PY{o}{.}\PY{n}{shape}\PY{p}{[}\PY{l+m+mi}{0}\PY{p}{]} \PY{k}{if} \PY{n}{n\PYZus{}components} \PY{o+ow}{is} \PY{k+kc}{None} \PY{k}{else} \PY{n}{n\PYZus{}components}
            \PY{k}{assert} \PY{n}{n\PYZus{}components} \PY{o}{\PYZlt{}}\PY{o}{=} \PY{n}{data}\PY{o}{.}\PY{n}{shape}\PY{p}{[}\PY{l+m+mi}{0}\PY{p}{]}\PY{p}{,} \PY{n}{f}\PY{l+s+s2}{\PYZdq{}}\PY{l+s+s2}{Got n\PYZus{}components larger than dimensionality of data!}\PY{l+s+s2}{\PYZdq{}}
            
            \PY{c+c1}{\PYZsh{} center the data}
            \PY{n}{data} \PY{o}{=} \PY{n}{data} \PY{o}{\PYZhy{}} \PY{n}{data}\PY{o}{.}\PY{n}{mean}\PY{p}{(}\PY{n}{axis} \PY{o}{=} \PY{l+m+mi}{1}\PY{p}{)}\PY{p}{[}\PY{p}{:}\PY{p}{,}\PY{n}{np}\PY{o}{.}\PY{n}{newaxis}\PY{p}{]}
            
            \PY{c+c1}{\PYZsh{} compute X times X transpose}
            \PY{n}{xxt} \PY{o}{=} \PY{n}{data}\PY{o}{.}\PY{n}{dot}\PY{p}{(}\PY{n}{data}\PY{o}{.}\PY{n}{T}\PY{p}{)}
            
            \PY{c+c1}{\PYZsh{} compute the eigenvectors and eigenvalues}
            \PY{n}{eig\PYZus{}val}\PY{p}{,} \PY{n}{eig\PYZus{}vec} \PY{o}{=} \PY{n}{np}\PY{o}{.}\PY{n}{linalg}\PY{o}{.}\PY{n}{eig}\PY{p}{(}\PY{n}{xxt}\PY{p}{)}
        
            \PY{c+c1}{\PYZsh{} sort the eigenvectors by eigenvalue and take the n\PYZus{}components largest ones}
            \PY{n+nb}{sorted} \PY{o}{=} \PY{n}{np}\PY{o}{.}\PY{n}{argsort}\PY{p}{(}\PY{n}{eig\PYZus{}val}\PY{p}{)}\PY{p}{[}\PY{p}{:}\PY{p}{:}\PY{o}{\PYZhy{}}\PY{l+m+mi}{1}\PY{p}{]}
        
            \PY{n}{eig\PYZus{}vec} \PY{o}{=} \PY{n}{eig\PYZus{}vec}\PY{p}{[}\PY{n+nb}{sorted}\PY{p}{,} \PY{p}{:}\PY{p}{]}
            \PY{n}{eig\PYZus{}val} \PY{o}{=} \PY{n}{eig\PYZus{}val}\PY{p}{[}\PY{n+nb}{sorted}\PY{p}{]}
        
            \PY{n}{components} \PY{o}{=} \PY{n}{eig\PYZus{}vec}\PY{p}{[}\PY{p}{:}\PY{n}{n\PYZus{}components}\PY{p}{]}
            
            \PY{c+c1}{\PYZsh{} compute X\PYZus{}projected, the projection of the data to the components}
            \PY{n}{X\PYZus{}projected} \PY{o}{=} \PY{n}{components}\PY{o}{.}\PY{n}{dot}\PY{p}{(}\PY{n}{data}\PY{p}{)}
        
            \PY{k}{return} \PY{n}{components}\PY{p}{,} \PY{n}{X\PYZus{}projected}  \PY{c+c1}{\PYZsh{} return the n\PYZus{}components first components and the pca projection of the data}
\end{Verbatim}


    \begin{Verbatim}[commandchars=\\\{\}]
{\color{incolor}In [{\color{incolor}4}]:} \PY{c+c1}{\PYZsh{} Example data to test your implementation }
        \PY{c+c1}{\PYZsh{} All the asserts on the bottom should go through if your implementation is correct}
        
        \PY{n}{data} \PY{o}{=} \PY{n}{np}\PY{o}{.}\PY{n}{array}\PY{p}{(}\PY{p}{[}
            \PY{p}{[} \PY{l+m+mi}{1}\PY{p}{,}  \PY{l+m+mi}{0}\PY{p}{,}  \PY{l+m+mi}{0}\PY{p}{,} \PY{o}{\PYZhy{}}\PY{l+m+mi}{1}\PY{p}{,}  \PY{l+m+mi}{0}\PY{p}{,}  \PY{l+m+mi}{0}\PY{p}{]}\PY{p}{,}
            \PY{p}{[} \PY{l+m+mi}{0}\PY{p}{,}  \PY{l+m+mi}{3}\PY{p}{,}  \PY{l+m+mi}{0}\PY{p}{,}  \PY{l+m+mi}{0}\PY{p}{,} \PY{o}{\PYZhy{}}\PY{l+m+mi}{3}\PY{p}{,}  \PY{l+m+mi}{0}\PY{p}{]}\PY{p}{,}
            \PY{p}{[} \PY{l+m+mi}{0}\PY{p}{,}  \PY{l+m+mi}{0}\PY{p}{,}  \PY{l+m+mi}{5}\PY{p}{,}  \PY{l+m+mi}{0}\PY{p}{,}  \PY{l+m+mi}{0}\PY{p}{,} \PY{o}{\PYZhy{}}\PY{l+m+mi}{5}\PY{p}{]}
        \PY{p}{]}\PY{p}{,} \PY{n}{dtype}\PY{o}{=}\PY{n}{np}\PY{o}{.}\PY{n}{float32}\PY{p}{)}
        
        \PY{c+c1}{\PYZsh{} add a random offset to all samples. it should not affect the results}
        \PY{n}{data} \PY{o}{+}\PY{o}{=} \PY{n}{np}\PY{o}{.}\PY{n}{random}\PY{o}{.}\PY{n}{randn}\PY{p}{(}\PY{n}{data}\PY{o}{.}\PY{n}{shape}\PY{p}{[}\PY{l+m+mi}{0}\PY{p}{]}\PY{p}{,} \PY{l+m+mi}{1}\PY{p}{)}
        
        \PY{n}{n\PYZus{}components} \PY{o}{=} \PY{l+m+mi}{2}
        \PY{n}{components}\PY{p}{,} \PY{n}{projection} \PY{o}{=} \PY{n}{pca}\PY{p}{(}\PY{n}{data}\PY{p}{,} \PY{n}{n\PYZus{}components}\PY{o}{=}\PY{n}{n\PYZus{}components}\PY{p}{)}  \PY{c+c1}{\PYZsh{} apply your implementation}
        
        \PY{c+c1}{\PYZsh{} the correct results are known (up to some signs)}
        \PY{n}{true\PYZus{}components} \PY{o}{=} \PY{n}{np}\PY{o}{.}\PY{n}{array}\PY{p}{(}\PY{p}{[}\PY{p}{[}\PY{l+m+mi}{0}\PY{p}{,} \PY{l+m+mi}{0}\PY{p}{,} \PY{l+m+mi}{1}\PY{p}{]}\PY{p}{,} \PY{p}{[}\PY{l+m+mi}{0}\PY{p}{,} \PY{l+m+mi}{1}\PY{p}{,} \PY{l+m+mi}{0}\PY{p}{]}\PY{p}{]}\PY{p}{,} \PY{n}{dtype}\PY{o}{=}\PY{n}{np}\PY{o}{.}\PY{n}{float32}\PY{p}{)}
        \PY{n}{true\PYZus{}projection} \PY{o}{=} \PY{n}{np}\PY{o}{.}\PY{n}{array}\PY{p}{(}\PY{p}{[}
            \PY{p}{[} \PY{l+m+mi}{0}\PY{p}{,}  \PY{l+m+mi}{0}\PY{p}{,}  \PY{l+m+mi}{5}\PY{p}{,}  \PY{l+m+mi}{0}\PY{p}{,}  \PY{l+m+mi}{0}\PY{p}{,} \PY{o}{\PYZhy{}}\PY{l+m+mi}{5}\PY{p}{]}\PY{p}{,}
            \PY{p}{[} \PY{l+m+mi}{0}\PY{p}{,}  \PY{l+m+mi}{3}\PY{p}{,}  \PY{l+m+mi}{0}\PY{p}{,}  \PY{l+m+mi}{0}\PY{p}{,} \PY{o}{\PYZhy{}}\PY{l+m+mi}{3}\PY{p}{,}  \PY{l+m+mi}{0}\PY{p}{]}
        \PY{p}{]}\PY{p}{,} \PY{n}{dtype}\PY{o}{=}\PY{n}{np}\PY{o}{.}\PY{n}{float32}\PY{p}{)}
        
        \PY{c+c1}{\PYZsh{} check that components match, up to sign}
        \PY{k}{assert} \PY{n+nb}{isinstance}\PY{p}{(}\PY{n}{components}\PY{p}{,} \PY{n}{np}\PY{o}{.}\PY{n}{ndarray}\PY{p}{)}\PY{p}{,} \PY{n}{f}\PY{l+s+s1}{\PYZsq{}}\PY{l+s+s1}{Expected components to be numpy array but got }\PY{l+s+s1}{\PYZob{}}\PY{l+s+s1}{type(components)\PYZcb{}}\PY{l+s+s1}{\PYZsq{}}
        \PY{k}{assert} \PY{n}{components}\PY{o}{.}\PY{n}{shape} \PY{o}{==} \PY{n}{true\PYZus{}components}\PY{o}{.}\PY{n}{shape}\PY{p}{,} \PY{n}{f}\PY{l+s+s1}{\PYZsq{}}\PY{l+s+si}{\PYZob{}components.shape\PYZcb{}}\PY{l+s+s1}{!=}\PY{l+s+si}{\PYZob{}true\PYZus{}components.shape\PYZcb{}}\PY{l+s+s1}{\PYZsq{}}
        \PY{k}{assert} \PY{n}{np}\PY{o}{.}\PY{n}{allclose}\PY{p}{(}\PY{n}{np}\PY{o}{.}\PY{n}{abs}\PY{p}{(}\PY{n}{components} \PY{o}{*} \PY{n}{true\PYZus{}components}\PY{p}{)}\PY{o}{.}\PY{n}{sum}\PY{p}{(}\PY{l+m+mi}{1}\PY{p}{)}\PY{p}{,} \PY{n}{np}\PY{o}{.}\PY{n}{ones}\PY{p}{(}\PY{n}{n\PYZus{}components}\PY{p}{)}\PY{p}{)}\PY{p}{,} \PY{n}{f}\PY{l+s+s1}{\PYZsq{}}\PY{l+s+s1}{Components not matching}\PY{l+s+s1}{\PYZsq{}}
        
        \PY{c+c1}{\PYZsh{} check that projections agree, taking into account potentially flipped components}
        \PY{k}{assert} \PY{n+nb}{isinstance}\PY{p}{(}\PY{n}{projection}\PY{p}{,} \PY{n}{np}\PY{o}{.}\PY{n}{ndarray}\PY{p}{)}\PY{p}{,} \PY{n}{f}\PY{l+s+s1}{\PYZsq{}}\PY{l+s+s1}{Expected projection to be numpy array but got }\PY{l+s+s1}{\PYZob{}}\PY{l+s+s1}{type(projection)\PYZcb{}}\PY{l+s+s1}{\PYZsq{}}
        \PY{k}{assert} \PY{n}{projection}\PY{o}{.}\PY{n}{shape} \PY{o}{==} \PY{p}{(}\PY{n}{n\PYZus{}components}\PY{p}{,} \PY{n}{data}\PY{o}{.}\PY{n}{shape}\PY{p}{[}\PY{l+m+mi}{1}\PY{p}{]}\PY{p}{)}\PY{p}{,} \PY{n}{f}\PY{l+s+s1}{\PYZsq{}}\PY{l+s+s1}{Incorrect shape of projection: Expected }\PY{l+s+s1}{\PYZob{}}\PY{l+s+s1}{(n\PYZus{}components, data.shape[1])\PYZcb{}, got }\PY{l+s+si}{\PYZob{}projection.shape\PYZcb{}}\PY{l+s+s1}{\PYZsq{}}
        \PY{k}{assert} \PY{n}{np}\PY{o}{.}\PY{n}{allclose}\PY{p}{(}\PY{n}{projection}\PY{p}{,} \PY{n}{true\PYZus{}projection} \PY{o}{*} \PY{p}{(}\PY{n}{components} \PY{o}{*} \PY{n}{true\PYZus{}components}\PY{p}{)}\PY{o}{.}\PY{n}{sum}\PY{p}{(}\PY{l+m+mi}{1}\PY{p}{,} \PY{n}{keepdims}\PY{o}{=}\PY{k+kc}{True}\PY{p}{)}\PY{p}{,} \PY{n}{atol}\PY{o}{=}\PY{l+m+mf}{1e\PYZhy{}6}\PY{p}{)}\PY{p}{,} \PY{n}{f}\PY{l+s+s1}{\PYZsq{}}\PY{l+s+s1}{Projections not matching}\PY{l+s+s1}{\PYZsq{}}
        
        \PY{n+nb}{print}\PY{p}{(}\PY{l+s+s1}{\PYZsq{}}\PY{l+s+s1}{Test successful!}\PY{l+s+s1}{\PYZsq{}}\PY{p}{)}
\end{Verbatim}


    \begin{Verbatim}[commandchars=\\\{\}]
Test successful!

    \end{Verbatim}

    \subsubsection{(b)}\label{b}

    Load the data (it is a subset of the data at
https://opendata.cern.ch/record/4910\#)

    \begin{Verbatim}[commandchars=\\\{\}]
{\color{incolor}In [{\color{incolor}8}]:} \PY{n}{features} \PY{o}{=} \PY{n}{np}\PY{o}{.}\PY{n}{load}\PY{p}{(}\PY{l+s+s1}{\PYZsq{}}\PY{l+s+s1}{data/dijet\PYZus{}features.npy}\PY{l+s+s1}{\PYZsq{}}\PY{p}{)}
        \PY{n}{labels} \PY{o}{=} \PY{n}{np}\PY{o}{.}\PY{n}{load}\PY{p}{(}\PY{l+s+s1}{\PYZsq{}}\PY{l+s+s1}{data/dijet\PYZus{}labels.npy}\PY{l+s+s1}{\PYZsq{}}\PY{p}{)}
        \PY{n}{label\PYZus{}names} \PY{o}{=} \PY{p}{[}\PY{l+s+s1}{\PYZsq{}}\PY{l+s+s1}{b}\PY{l+s+s1}{\PYZsq{}}\PY{p}{,} \PY{l+s+s1}{\PYZsq{}}\PY{l+s+s1}{c}\PY{l+s+s1}{\PYZsq{}}\PY{p}{,} \PY{l+s+s1}{\PYZsq{}}\PY{l+s+s1}{q}\PY{l+s+s1}{\PYZsq{}}\PY{p}{]}  \PY{c+c1}{\PYZsh{} bottom, charm or light quarks}
        
        \PY{n+nb}{print}\PY{p}{(}\PY{n}{f}\PY{l+s+s1}{\PYZsq{}}\PY{l+s+s1}{\PYZob{}}\PY{l+s+s1}{features.shape=\PYZcb{}, }\PY{l+s+s1}{\PYZob{}}\PY{l+s+s1}{labels.shape=\PYZcb{}}\PY{l+s+s1}{\PYZsq{}}\PY{p}{)}  \PY{c+c1}{\PYZsh{} print the shapes}
        
        \PY{c+c1}{\PYZsh{} TODO: print how many samples of each class are present in the data (hint: numpy.unique)}
        \PY{n}{\PYZus{}labels}\PY{p}{,} \PY{n}{\PYZus{}counts} \PY{o}{=} \PY{n}{np}\PY{o}{.}\PY{n}{unique}\PY{p}{(}\PY{n}{labels}\PY{p}{,} \PY{n}{return\PYZus{}counts} \PY{o}{=} \PY{k+kc}{True}\PY{p}{)}
        
        \PY{k}{for} \PY{n}{label}\PY{p}{,} \PY{n}{count} \PY{o+ow}{in} \PY{n+nb}{zip}\PY{p}{(}\PY{n}{\PYZus{}labels}\PY{p}{,} \PY{n}{\PYZus{}counts}\PY{p}{)}\PY{p}{:}
            \PY{n+nb}{print}\PY{p}{(}\PY{n}{f}\PY{l+s+s1}{\PYZsq{}}\PY{l+s+s1}{Number of instaces with label }\PY{l+s+si}{\PYZob{}label\PYZcb{}}\PY{l+s+s1}{: }\PY{l+s+si}{\PYZob{}count\PYZcb{}}\PY{l+s+s1}{\PYZsq{}}\PY{p}{)}
\end{Verbatim}


    \begin{Verbatim}[commandchars=\\\{\}]
features.shape=(116, 2233), labels.shape=(2233,)
Number of instaces with label 0.0: 999
Number of instaces with label 1.0: 864
Number of instaces with label 2.0: 370

    \end{Verbatim}

    Normalize the data

    \begin{Verbatim}[commandchars=\\\{\}]
{\color{incolor}In [{\color{incolor}70}]:} \PY{c+c1}{\PYZsh{} TODO: subtract the mean of each feature to center the data}
         \PY{n}{features} \PY{o}{=} \PY{n}{features} \PY{o}{\PYZhy{}} \PY{n}{features}\PY{o}{.}\PY{n}{mean}\PY{p}{(}\PY{n}{axis} \PY{o}{=} \PY{l+m+mi}{1}\PY{p}{)}\PY{p}{[}\PY{p}{:}\PY{p}{,}\PY{n}{np}\PY{o}{.}\PY{n}{newaxis}\PY{p}{]}
         
         \PY{c+c1}{\PYZsh{} TODO: divide by the standard deviation of each feature to normalize the variance}
         \PY{n}{features} \PY{o}{=} \PY{n}{features} \PY{o}{/} \PY{n}{features}\PY{o}{.}\PY{n}{std}\PY{p}{(}\PY{n}{axis} \PY{o}{=} \PY{l+m+mi}{1}\PY{p}{)}\PY{p}{[}\PY{p}{:}\PY{p}{,}\PY{n}{np}\PY{o}{.}\PY{n}{newaxis}\PY{p}{]}
\end{Verbatim}


    \subsubsection{(c)}\label{c}

Compute a 2D PCA projection and make a scatterplot of the result, once
without color, once coloring the dots by label

    \begin{Verbatim}[commandchars=\\\{\}]
{\color{incolor}In [{\color{incolor}42}]:} \PY{c+c1}{\PYZsh{} TODO: apply PCA as implemented in (a)}
         \PY{n}{components}\PY{p}{,} \PY{n}{projected\PYZus{}features} \PY{o}{=} \PY{n}{pca}\PY{p}{(}\PY{n}{data}\PY{o}{=}\PY{n}{features}\PY{p}{,} \PY{n}{n\PYZus{}components}\PY{o}{=}\PY{l+m+mi}{2}\PY{p}{)}
\end{Verbatim}


    \begin{Verbatim}[commandchars=\\\{\}]
{\color{incolor}In [{\color{incolor}53}]:} \PY{l+s+sd}{\PYZsq{}\PYZsq{}\PYZsq{}The pca returns projected features with an imaginary value. }
         \PY{l+s+sd}{As we can see, this value is very small (at least for the first two components) in magnitude\PYZsq{}\PYZsq{}\PYZsq{}}
         \PY{n}{np}\PY{o}{.}\PY{n}{abs}\PY{p}{(}\PY{n}{np}\PY{o}{.}\PY{n}{imag}\PY{p}{(}\PY{n}{projected\PYZus{}features}\PY{p}{)}\PY{p}{)}\PY{o}{.}\PY{n}{max}\PY{p}{(}\PY{p}{)}
\end{Verbatim}


\begin{Verbatim}[commandchars=\\\{\}]
{\color{outcolor}Out[{\color{outcolor}53}]:} 5.267533154478281e-09
\end{Verbatim}
            
    \begin{Verbatim}[commandchars=\\\{\}]
{\color{incolor}In [{\color{incolor}54}]:} \PY{c+c1}{\PYZsh{} TODO: make a scatterplot of the PCA projection}
         \PY{n}{plt}\PY{o}{.}\PY{n}{scatter}\PY{p}{(}\PY{o}{*}\PY{n}{np}\PY{o}{.}\PY{n}{real}\PY{p}{(}\PY{n}{projected\PYZus{}features}\PY{p}{)}\PY{p}{)}
\end{Verbatim}


\begin{Verbatim}[commandchars=\\\{\}]
{\color{outcolor}Out[{\color{outcolor}54}]:} <matplotlib.collections.PathCollection at 0x7fc04ca47430>
\end{Verbatim}
            
    \begin{center}
    \adjustimage{max size={0.9\linewidth}{0.9\paperheight}}{output_13_1.png}
    \end{center}
    { \hspace*{\fill} \\}
    
    \begin{Verbatim}[commandchars=\\\{\}]
{\color{incolor}In [{\color{incolor}66}]:} \PY{c+c1}{\PYZsh{} TODO: make a scatterplot, coloring the dots by their label and including a legend with the label names}
         \PY{c+c1}{\PYZsh{} (hint: one way is to call plt.scatter once for each of the three possible labels)}
         \PY{n}{plt}\PY{o}{.}\PY{n}{figure}\PY{p}{(}\PY{n}{figsize} \PY{o}{=} \PY{p}{(}\PY{l+m+mi}{7}\PY{p}{,}\PY{l+m+mi}{7}\PY{p}{)}\PY{p}{)}
         \PY{n}{plt}\PY{o}{.}\PY{n}{title}\PY{p}{(}\PY{l+s+s1}{\PYZsq{}}\PY{l+s+s1}{LHCb data \PYZhy{} projected onto two pricipal components}\PY{l+s+s1}{\PYZsq{}}\PY{p}{)}
         
         \PY{k}{for} \PY{n}{label} \PY{o+ow}{in} \PY{n}{np}\PY{o}{.}\PY{n}{unique}\PY{p}{(}\PY{n}{labels}\PY{p}{)}\PY{p}{:}
             \PY{n}{plt}\PY{o}{.}\PY{n}{scatter}\PY{p}{(}\PY{o}{*}\PY{n}{np}\PY{o}{.}\PY{n}{real}\PY{p}{(}\PY{n}{projected\PYZus{}features}\PY{p}{[}\PY{p}{:}\PY{p}{,}\PY{n}{labels} \PY{o}{==} \PY{n}{label}\PY{p}{]}\PY{p}{)}\PY{p}{,} \PY{n}{marker} \PY{o}{=} \PY{l+s+s1}{\PYZsq{}}\PY{l+s+s1}{.}\PY{l+s+s1}{\PYZsq{}}\PY{p}{,} \PY{n}{label} \PY{o}{=} \PY{n}{label}\PY{p}{)}
         
         \PY{n}{plt}\PY{o}{.}\PY{n}{legend}\PY{p}{(}\PY{p}{)}
         
         \PY{n}{plt}\PY{o}{.}\PY{n}{xlabel}\PY{p}{(}\PY{l+s+s1}{\PYZsq{}}\PY{l+s+s1}{pc 1}\PY{l+s+s1}{\PYZsq{}}\PY{p}{)}
         \PY{n}{plt}\PY{o}{.}\PY{n}{ylabel}\PY{p}{(}\PY{l+s+s1}{\PYZsq{}}\PY{l+s+s1}{pc 2}\PY{l+s+s1}{\PYZsq{}}\PY{p}{)}
\end{Verbatim}


\begin{Verbatim}[commandchars=\\\{\}]
{\color{outcolor}Out[{\color{outcolor}66}]:} Text(0, 0.5, 'pc 2')
\end{Verbatim}
            
    \begin{center}
    \adjustimage{max size={0.9\linewidth}{0.9\paperheight}}{output_14_1.png}
    \end{center}
    { \hspace*{\fill} \\}
    
    \subsection{2 Nonlinear Dimension
Reduction}\label{nonlinear-dimension-reduction}

    \begin{Verbatim}[commandchars=\\\{\}]
{\color{incolor}In [{\color{incolor}67}]:} \PY{k+kn}{import} \PY{n+nn}{umap}  \PY{c+c1}{\PYZsh{} import umap\PYZhy{}learn, see https://umap\PYZhy{}learn.readthedocs.io/}
\end{Verbatim}


    \begin{Verbatim}[commandchars=\\\{\}]
{\color{incolor}In [{\color{incolor}84}]:} \PY{c+c1}{\PYZsh{} if you have not done 1(b) yet, you can load the normalized features directly:}
         \PY{n}{features} \PY{o}{=} \PY{n}{np}\PY{o}{.}\PY{n}{load}\PY{p}{(}\PY{l+s+s1}{\PYZsq{}}\PY{l+s+s1}{data/dijet\PYZus{}features\PYZus{}normalized.npy}\PY{l+s+s1}{\PYZsq{}}\PY{p}{)}
         \PY{n}{labels} \PY{o}{=} \PY{n}{np}\PY{o}{.}\PY{n}{load}\PY{p}{(}\PY{l+s+s1}{\PYZsq{}}\PY{l+s+s1}{data/dijet\PYZus{}labels.npy}\PY{l+s+s1}{\PYZsq{}}\PY{p}{)}
         \PY{n}{label\PYZus{}names} \PY{o}{=} \PY{p}{[}\PY{l+s+s1}{\PYZsq{}}\PY{l+s+s1}{b}\PY{l+s+s1}{\PYZsq{}}\PY{p}{,} \PY{l+s+s1}{\PYZsq{}}\PY{l+s+s1}{c}\PY{l+s+s1}{\PYZsq{}}\PY{p}{,} \PY{l+s+s1}{\PYZsq{}}\PY{l+s+s1}{q}\PY{l+s+s1}{\PYZsq{}}\PY{p}{]}  \PY{c+c1}{\PYZsh{} bottom, charm or light quarks}
\end{Verbatim}


    \subsubsection{(a)}\label{a}

    \begin{Verbatim}[commandchars=\\\{\}]
{\color{incolor}In [{\color{incolor}89}]:} \PY{c+c1}{\PYZsh{} TODO: Apply umap on the normalized jet features from excercise 1. It will take a couple of seconds.}
         \PY{c+c1}{\PYZsh{} note: umap uses a different convention regarding the feature\PYZhy{} and sample dimension, N x p instead of p x N!}
         \PY{n}{reducer} \PY{o}{=} \PY{n}{umap}\PY{o}{.}\PY{n}{UMAP}\PY{p}{(}\PY{p}{)}
         
         \PY{n}{embedding} \PY{o}{=} \PY{n}{reducer}\PY{o}{.}\PY{n}{fit\PYZus{}transform}\PY{p}{(}\PY{n}{features}\PY{o}{.}\PY{n}{T}\PY{p}{)}
\end{Verbatim}


    \begin{Verbatim}[commandchars=\\\{\}]
{\color{incolor}In [{\color{incolor}99}]:} \PY{c+c1}{\PYZsh{} TODO: make a scatterplot of the UMAP projection}
         \PY{c+c1}{\PYZsh{}plt.figure()}
         \PY{c+c1}{\PYZsh{}plt.scatter(embedding[:,0], embedding[:,1])}
         
         \PY{c+c1}{\PYZsh{} TODO: make a scatterplot, coloring the dots by their label and including a legend with the label names}
         \PY{c+c1}{\PYZsh{} (hint: one way is to call plt.scatter once for each of the three possible labels)}
         
         \PY{k}{def} \PY{n+nf}{plot\PYZus{}umap}\PY{p}{(}\PY{n}{embedding}\PY{p}{)}\PY{p}{:}
             \PY{n}{plt}\PY{o}{.}\PY{n}{figure}\PY{p}{(}\PY{n}{figsize} \PY{o}{=} \PY{p}{(}\PY{l+m+mi}{5}\PY{p}{,}\PY{l+m+mi}{5}\PY{p}{)}\PY{p}{)}
             \PY{n}{plt}\PY{o}{.}\PY{n}{title}\PY{p}{(}\PY{l+s+s1}{\PYZsq{}}\PY{l+s+s1}{LHCb data \PYZhy{} projected onto two dimensions using UMAP}\PY{l+s+s1}{\PYZsq{}}\PY{p}{)}
         
             \PY{k}{for} \PY{n}{label} \PY{o+ow}{in} \PY{n}{np}\PY{o}{.}\PY{n}{unique}\PY{p}{(}\PY{n}{labels}\PY{p}{)}\PY{p}{:}
                 \PY{n}{plt}\PY{o}{.}\PY{n}{scatter}\PY{p}{(}\PY{o}{*}\PY{n}{embedding}\PY{o}{.}\PY{n}{T}\PY{p}{[}\PY{p}{:}\PY{p}{,}\PY{n}{labels} \PY{o}{==} \PY{n}{label}\PY{p}{]}\PY{p}{,} \PY{n}{marker} \PY{o}{=} \PY{l+s+s1}{\PYZsq{}}\PY{l+s+s1}{.}\PY{l+s+s1}{\PYZsq{}}\PY{p}{,} 
                             \PY{n}{label} \PY{o}{=} \PY{n+nb}{dict}\PY{p}{(}\PY{n+nb}{zip}\PY{p}{(}\PY{p}{[}\PY{l+m+mf}{0.0}\PY{p}{,} \PY{l+m+mf}{1.0}\PY{p}{,} \PY{l+m+mf}{2.0}\PY{p}{]}\PY{p}{,} \PY{n}{label\PYZus{}names}\PY{p}{)}\PY{p}{)}\PY{p}{[}\PY{n}{label}\PY{p}{]}\PY{p}{)}
         
             \PY{n}{plt}\PY{o}{.}\PY{n}{legend}\PY{p}{(}\PY{p}{)}
         
             \PY{n}{plt}\PY{o}{.}\PY{n}{xlabel}\PY{p}{(}\PY{l+s+s1}{\PYZsq{}}\PY{l+s+s1}{dim 1}\PY{l+s+s1}{\PYZsq{}}\PY{p}{)}
             \PY{n}{plt}\PY{o}{.}\PY{n}{ylabel}\PY{p}{(}\PY{l+s+s1}{\PYZsq{}}\PY{l+s+s1}{dim 2}\PY{l+s+s1}{\PYZsq{}}\PY{p}{)}
         
             \PY{n}{plt}\PY{o}{.}\PY{n}{show}\PY{p}{(}\PY{p}{)}
         
         \PY{n}{plot\PYZus{}umap}\PY{p}{(}\PY{n}{embedding}\PY{p}{)}
\end{Verbatim}


    \begin{center}
    \adjustimage{max size={0.9\linewidth}{0.9\paperheight}}{output_20_0.png}
    \end{center}
    { \hspace*{\fill} \\}
    
    \subsubsection{(b)}\label{b}

    \begin{Verbatim}[commandchars=\\\{\}]
{\color{incolor}In [{\color{incolor}98}]:} \PY{k}{for} \PY{n}{n\PYZus{}neighbors} \PY{o+ow}{in} \PY{p}{(}\PY{l+m+mi}{2}\PY{p}{,} \PY{l+m+mi}{4}\PY{p}{,} \PY{l+m+mi}{8}\PY{p}{,} \PY{l+m+mi}{15}\PY{p}{,} \PY{l+m+mi}{30}\PY{p}{,} \PY{l+m+mi}{60}\PY{p}{,} \PY{l+m+mi}{100}\PY{p}{)}\PY{p}{:}
             \PY{c+c1}{\PYZsh{} TODO: repeat the above, varying the n\PYZus{}neighbors parameter of UMAP}
             \PY{n+nb}{print}\PY{p}{(}\PY{n}{f}\PY{l+s+s1}{\PYZsq{}}\PY{l+s+s1}{Number of neighbours used in UMAP }\PY{l+s+si}{\PYZob{}n\PYZus{}neighbors\PYZcb{}}\PY{l+s+s1}{\PYZsq{}}\PY{p}{)}
         
             \PY{n}{reducer} \PY{o}{=} \PY{n}{umap}\PY{o}{.}\PY{n}{UMAP}\PY{p}{(}\PY{n}{n\PYZus{}neighbors}\PY{o}{=}\PY{n}{n\PYZus{}neighbors}\PY{p}{)}
             
             \PY{n}{embedding} \PY{o}{=} \PY{n}{reducer}\PY{o}{.}\PY{n}{fit\PYZus{}transform}\PY{p}{(}\PY{n}{features}\PY{o}{.}\PY{n}{T}\PY{p}{)}
         
             \PY{n}{plot\PYZus{}umap}\PY{p}{(}\PY{n}{embedding}\PY{p}{)}
\end{Verbatim}


    \begin{Verbatim}[commandchars=\\\{\}]
Starting with 2

    \end{Verbatim}

    \begin{Verbatim}[commandchars=\\\{\}]
/home/paul\_zuern/miniconda3/envs/mlph/lib/python3.9/site-packages/umap/spectral.py:260: UserWarning: WARNING: spectral initialisation failed! The eigenvector solver
failed. This is likely due to too small an eigengap. Consider
adding some noise or jitter to your data.

Falling back to random initialisation!
  warn(
/home/paul\_zuern/miniconda3/envs/mlph/lib/python3.9/site-packages/umap/spectral.py:260: UserWarning: WARNING: spectral initialisation failed! The eigenvector solver
failed. This is likely due to too small an eigengap. Consider
adding some noise or jitter to your data.

Falling back to random initialisation!
  warn(

    \end{Verbatim}

    \begin{Verbatim}[commandchars=\\\{\}]
Starting with 4
Starting with 8
Starting with 15
Starting with 30
Starting with 60
Starting with 100

    \end{Verbatim}

    \begin{center}
    \adjustimage{max size={0.9\linewidth}{0.9\paperheight}}{output_22_3.png}
    \end{center}
    { \hspace*{\fill} \\}
    
    \begin{center}
    \adjustimage{max size={0.9\linewidth}{0.9\paperheight}}{output_22_4.png}
    \end{center}
    { \hspace*{\fill} \\}
    
    \begin{center}
    \adjustimage{max size={0.9\linewidth}{0.9\paperheight}}{output_22_5.png}
    \end{center}
    { \hspace*{\fill} \\}
    
    \begin{center}
    \adjustimage{max size={0.9\linewidth}{0.9\paperheight}}{output_22_6.png}
    \end{center}
    { \hspace*{\fill} \\}
    
    \begin{center}
    \adjustimage{max size={0.9\linewidth}{0.9\paperheight}}{output_22_7.png}
    \end{center}
    { \hspace*{\fill} \\}
    
    \begin{center}
    \adjustimage{max size={0.9\linewidth}{0.9\paperheight}}{output_22_8.png}
    \end{center}
    { \hspace*{\fill} \\}
    
    \begin{center}
    \adjustimage{max size={0.9\linewidth}{0.9\paperheight}}{output_22_9.png}
    \end{center}
    { \hspace*{\fill} \\}
    
    \subsubsection{(c)}\label{c}

    \begin{Verbatim}[commandchars=\\\{\}]
{\color{incolor}In [{\color{incolor}101}]:} \PY{k+kn}{from} \PY{n+nn}{sklearn}\PY{n+nn}{.}\PY{n+nn}{decomposition} \PY{k}{import} \PY{n}{PCA} \PY{k}{as} \PY{n}{skPCA}
          
          \PY{k}{for} \PY{n}{n\PYZus{}components} \PY{o+ow}{in} \PY{p}{(}\PY{l+m+mi}{2}\PY{p}{,} \PY{l+m+mi}{4}\PY{p}{,} \PY{l+m+mi}{8}\PY{p}{,} \PY{l+m+mi}{16}\PY{p}{,} \PY{l+m+mi}{32}\PY{p}{,} \PY{l+m+mi}{64}\PY{p}{,} \PY{n+nb}{len}\PY{p}{(}\PY{n}{features}\PY{p}{)}\PY{p}{)}\PY{p}{:}
              \PY{c+c1}{\PYZsh{} TODO: project to the n\PYZhy{}components first principal components }
              \PY{c+c1}{\PYZsh{}       (use your implementation from ex. 1 or PCA from scikit\PYZhy{}learn)}
              \PY{n}{PCA} \PY{o}{=} \PY{n}{skPCA}\PY{p}{(}\PY{n}{n\PYZus{}components}\PY{o}{=}\PY{n}{n\PYZus{}components}\PY{p}{)}
              
              \PY{n}{projected\PYZus{}features} \PY{o}{=} \PY{n}{PCA}\PY{o}{.}\PY{n}{fit\PYZus{}transform}\PY{p}{(}\PY{n}{features}\PY{o}{.}\PY{n}{T}\PY{p}{)}
          
              \PY{c+c1}{\PYZsh{} TODO: apply UMAP to get from n\PYZus{}components to just two dimensions}
              \PY{n}{reducer} \PY{o}{=} \PY{n}{umap}\PY{o}{.}\PY{n}{UMAP}\PY{p}{(}\PY{p}{)}
          
              \PY{n}{embedding} \PY{o}{=} \PY{n}{reducer}\PY{o}{.}\PY{n}{fit\PYZus{}transform}\PY{p}{(}\PY{n}{projected\PYZus{}features}\PY{p}{)}
          
              \PY{c+c1}{\PYZsh{} TODO: again, make scatterplots as before}
              \PY{n}{plot\PYZus{}umap}\PY{p}{(}\PY{n}{embedding}\PY{p}{)}
\end{Verbatim}


    \begin{center}
    \adjustimage{max size={0.9\linewidth}{0.9\paperheight}}{output_24_0.png}
    \end{center}
    { \hspace*{\fill} \\}
    
    \begin{center}
    \adjustimage{max size={0.9\linewidth}{0.9\paperheight}}{output_24_1.png}
    \end{center}
    { \hspace*{\fill} \\}
    
    \begin{center}
    \adjustimage{max size={0.9\linewidth}{0.9\paperheight}}{output_24_2.png}
    \end{center}
    { \hspace*{\fill} \\}
    
    \begin{center}
    \adjustimage{max size={0.9\linewidth}{0.9\paperheight}}{output_24_3.png}
    \end{center}
    { \hspace*{\fill} \\}
    
    \begin{center}
    \adjustimage{max size={0.9\linewidth}{0.9\paperheight}}{output_24_4.png}
    \end{center}
    { \hspace*{\fill} \\}
    
    \begin{center}
    \adjustimage{max size={0.9\linewidth}{0.9\paperheight}}{output_24_5.png}
    \end{center}
    { \hspace*{\fill} \\}
    
    \begin{center}
    \adjustimage{max size={0.9\linewidth}{0.9\paperheight}}{output_24_6.png}
    \end{center}
    { \hspace*{\fill} \\}
    

    % Add a bibliography block to the postdoc
    
    
    
    \end{document}
